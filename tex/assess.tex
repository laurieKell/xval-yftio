
There has been a recent trend in stock assessment to use integrated analysis that combines several sources of data into a single model by a joint likelihood for the observed data (Doubleday, 1976, Fournier and Archibald, 1982, Maunder and Punt, 2013).  Datasets include records of catches and landings, indices of abundance based on catch per unit (CPUE) and from research surveys, and the length classes and/or ages compositions based on samples. A commonly used integrated assessment method is Stock Synthesis (Methot and Wetzel, 2013) that can be configured in multiple ways e.g. SSS and ASPM.

Alternatives are biomass dynamic models based on a surplus production function that requires the estimation or fixing of fewer parameters, since many parameters required in integrated assessments are difficult to estimate in practice (e.g. Lee et al.). Once such example is JABBA an open source package that presents a unifying, flexible framework for biomass dynamic modelling, runs quickly, and generates https://www.overleaf.com/project/5ca9fed01e2a625dbaeae31freproducible stock status estimates.

A problem is how to compare these different models


\begin{description}
\item{SS} 
Stock Synthesis (Methot and Wetzel 2013; SS) is widely used to perform integrated assessments for fish stocks in the United States and throughout the world. While Stock Synthesis is typically applied in data-rich circumstances, its use in data-moderate stock assessments (where catch time series and abundance indices are available, and catch composition data is limited or absent) has significantly increased in recent years. The uptake of Stock Synthesis for integrated data-moderate stock assessments has occurred especially in assessments developed for Regional Fishery Management Organizations (RFMOs). Various stocks of billfish and pelagic sharks that were historically assessed using Surplus Production Models and Virtual Population Analysis (VPA) are now moving towards full implementation in Stock Synthesis as additional data become available (e.g. Courtney et al. 2017; Schirripa 2019). Similarly, there has been a recent increase in Stock Synthesis benchmark assessment in Europe in place of the conventionally used VPA with extended survivor analysis (XSA) in the region (Max Refs). The visualization of model outputs and implementation of diagnostics for Stock Synthesis is facilitated by the availability of a comprehensive collection of R functions (r4ss; https://github.com/r4ss).

\item{ASPM} 
Maunder and Piner (2015) proposed the ASPM age-structured production model (ASPM) as a diagnostic of process dynamics. This diagnostic evaluates whether the observed catches alone (taken out of approximately the correct ages) can explain trends in the index of abundance. On the one hand, Maunder and Piner (2015) suggest that if the ASPM is able to fit well to the indices of abundance that have good contrast (i.e. those that have declining as well as increasing trends), the production function likely exists, and the indices will provide information about absolute abundance. On the other hand, the authors suggest that if there is not a good fit to the indices, then the catch data alone cannot explain the trajectories depicted in the indices of relative abundance. This can have several causes: (i) the stock is recruitment-driven; (ii) the stock has not yet declined to the point at which catch is a major factor influencing abundance; (iii) the base-case model is incorrect; or (iv) the indices of relative abundance are not proportional to abundance. Alternatively, failure in the ASPM may indicate a system that is not well organized (e.g., stock structures or data are incorrect) so that a real fishing signal is lost or where unknown environmental drivers control population abundance. The ASPM was shown via simulation analyses to be the only tested diagnostic capable of detecting misspecification of the key systems-modeled processes that control the shape of the production function (Carvalho et al., 2017).
\item{Jabba}
This study presents a new, open-source modelling software entitled ‘Just Another Bayesian Biomass Assessment’ (JABBA). JABBA can be used for biomass dynamic stock assessment applications, and has emerged from the development of a Bayesian State-Space Surplus Production Model framework, already applied in stock assessments of sharks, tuna, and billfishes around the world. JABBA presents a unifying, flexible framework for biomass dynamic modelling, runs quickly, and generates reproducible stock status estimates and diagnostic tools. Specific emphasis has been placed on flexibility for specifying alternative scenarios, achieving high stability and improved convergence rates. Default JABBA features include: 1) an integrated state-space tool for averaging and automatically fitting multiple catch per unit effort (CPUE) time series; 2) data-weighting through estimation of additional observation variance for individual or grouped CPUE; 3) selection of Fox, Schaefer, or Pella-Tomlinson production functions; 4) options to fix or estimate process and observation variance components; 5) model diagnostic tools; 6) future projections for alternative catch regimes; and 7) a suite of inbuilt graphics illustrating model fit diagnostics and stock status results. As a case study, JABBA is applied to the 2017 assessment input data for South Atlantic swordfish (Xiphias gladius). We envision that JABBA will become a widely used, open-source stock assessment tool, readily improved and modified by the global scientific community.
\end{description}